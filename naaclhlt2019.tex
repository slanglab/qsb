%
% File naacl2019.tex
%
%% Based on the style files for ACL 2018 and NAACL 2018, which were
%% Based on the style files for ACL-2015, with some improvements
%%  taken from the NAACL-2016 style
%% Based on the style files for ACL-2014, which were, in turn,
%% based on ACL-2013, ACL-2012, ACL-2011, ACL-2010, ACL-IJCNLP-2009,
%% EACL-2009, IJCNLP-2008...
%% Based on the style files for EACL 2006 by 
%%e.agirre@ehu.es or Sergi.Balari@uab.es
%% and that of ACL 08 by Joakim Nivre and Noah Smith

\documentclass[11pt,a4paper]{article}
\usepackage[hyperref]{naaclhlt2019}
\usepackage{times}
\usepackage{latexsym}

\usepackage{url}

%\aclfinalcopy % Uncomment this line for the final submission
%\def\aclpaperid{***} %  Enter the acl Paper ID here

%\setlength\titlebox{5cm}
% You can expand the titlebox if you need extra space
% to show all the authors. Please do not make the titlebox
% smaller than 5cm (the original size); we will check this
% in the camera-ready version and ask you to change it back.

\newcommand\BibTeX{B{\sc ib}\TeX}

\title{Extractive sentence compression under lexical and length constraints}

\author{First Author \\
  Affiliation / Address line 1 \\
  Affiliation / Address line 2 \\
  Affiliation / Address line 3 \\
  {\tt email@domain} \\\And
  Second Author \\
  Affiliation / Address line 1 \\
  Affiliation / Address line 2 \\
  Affiliation / Address line 3 \\
  {\tt email@domain} \\}

\date{}

\begin{document}
\maketitle

\begin{abstract}
TODO
\end{abstract}

\section{Outline}

\begin{enumerate}
\item{$(q,s,r)$ compression is useful.}
\item{How do we do it: ILPs, neural net taggers, iterative deletion supervised w/ something. (\S\ref{s:method}) Neural net taggers are a research question.}
\item{Why iterative deletion is good. Description of current system for making deletion decisions w/ human supervision. Possibly make this system better too}
\item{Computational experiments (1): Properties of q,s,r compression in English. }
\item{Computational experiments (2): F and A data, interpreted as q,s,r supervision/data. Enumerate methods and bold the one with the best token-level F1 score.}
\item{Possible post hoc human experiment? Is our system actually good?}
\end{enumerate}




\section{Why iterative deletion is good}\label{s:method}
\begin{enumerate}
\item{Historical reasons}
\item{practical}
\item{grammatical bias + syntax bias. Leverages a lot that is know from linguistics: e.g. adjuncts, transitive bias, function words.}
\item{budget and query are trivial}
\item{linear (taggers) vs. quadratic (this work) vs. ILPs (exponential). }
\item{single op supervision can be used}
\item{avoid computational waste w/ grammar! HUGE divergence between mathematical properties of compression problem (exponential) and ways you can actually compress a sentence.}
\item{Cool idea: you could do a neural system that decides when/what to delete, using F and A as supervision? Each tree gets an encoding and you do a softmax over trees? This sounds hard to implement.}
\end{enumerate}

\subsection{Preprocessing with extract}
Extract preprocessing. Make up a few simple rules. They will help a bunch. Only like 6 dependency types work at all for extract. This is in realfake/createdata/extract at the moment.

\section{Computational experiments part 1: investigating properties of q,s,r compression}
\begin{enumerate}
\item{q = a list of 1 to 3 NER}
\item{r = random}
\item{What is the size of the minimum compression?}
\item{Reachability by position of q in syntax tree. if q is more than one entity then how the entities are dispersed across the tree probably matters a bunch too}
\item{avoid computational waste w/ grammar.  ILPs worry about pruning the nsubj. You could do this w/ a plot: ops saved v. p(deletion endorsement). Some ops get rid of lots of tokens w/ very high probability of deletion endorsements: e.g. parataxis, a great op. constrast: pruning a noun subj does not  ever work and usually does not delete many tokens.}
\item{what is the empirical number of ops (i.e. prune decisions) if you greedily drop branches but never drop if single op probability is less than $p$. My gut is you can make this problem way, way, simpler than implied by exponential formulation}
\item{position of query in tree vs. reachability at $r$}
\item{distribution of number of ops used for different q and r: when choosing ops at random? when choosing greedily? when using based on p(endorsement)?}
\item{Other stuff: min compression, reachability, operations saved w/ big prunes? position of query in the tree}
\end{enumerate}

\section{Computational experiments part 2: sentence compression}
Interpret F and A as Q,S,R supervision/data. Then test methods which do this. Measure w/ token level F1


\begin{table}[htb!]
\begin{tabular}{ll}
\centering
Approach & F1 \\ \hline
\textbf{This work}         & \textbf{A}           \\
F and A (2013)    & B           \\
C and L (2008)    & C          
\end{tabular}
\end{table}

\begin{enumerate}
\item{q = a list of 1 to 3 NER, generated at random. If the compression has no NER, use noun unigrams I guess? or probably just skip it}
\item{r = length of gold}
\end{enumerate}



\section{TODO}
\begin{enumerate}
\item{Look at query-biased snippets in IR. There is an old email about this. Look at J and M chapter 23 second edition. Ask Hamed and John}
\item{Find old ILP code and get stuff working}
\item{Brendan email about subsective adjectives ellie pavlik work. How will we deal with semantics? }
\end{enumerate}

\section{Future work}
Constrained neural network methods that actually do pruning w/ tree like deletion structures. 
Real-fake detection?




\end{document}
