%
% File naacl2019.tex
%
%% Based on the style files for ACL 2018 and NAACL 2018, which were
%% Based on the style files for ACL-2015, with some improvements
%%  taken from the NAACL-2016 style
%% Based on the style files for ACL-2014, which were, in turn,
%% based on ACL-2013, ACL-2012, ACL-2011, ACL-2010, ACL-IJCNLP-2009,
%% EACL-2009, IJCNLP-2008...
%% Based on the style files for EACL 2006 by 
%%e.agirre@ehu.es or Sergi.Balari@uab.es
%% and that of ACL 08 by Joakim Nivre and Noah Smith

\documentclass[11pt,a4paper]{article}
\usepackage[hyperref]{naaclhlt2019}
\usepackage{times}
\usepackage{latexsym}

\newcommand{\ahcomment}[1]{\textcolor{blue}{[#1 -AH]}}

\usepackage{url}

%\aclfinalcopy % Uncomment this line for the final submission
%\def\aclpaperid{***} %  Enter the acl Paper ID here

%\setlength\titlebox{5cm}
% You can expand the titlebox if you need extra space
% to show all the authors. Please do not make the titlebox
% smaller than 5cm (the original size); we will check this
% in the camera-ready version and ask you to change it back.

\newcommand\BibTeX{B{\sc ib}\TeX}

\title{Extractive sentence compression under lexical and length constraints}

\author{First Author \\
  Affiliation / Address line 1 \\
  Affiliation / Address line 2 \\
  Affiliation / Address line 3 \\
  {\tt email@domain} \\\And
  Second Author \\
  Affiliation / Address line 1 \\
  Affiliation / Address line 2 \\
  Affiliation / Address line 3 \\
  {\tt email@domain} \\}

\date{}

\begin{document}
\maketitle

\begin{abstract}
Traditional approaches to extractive sentence compression seek to reduce the length of a sentence, while retaining the most ``important'' information from the source. But query-focused applications such as document search engines or exploratory search interfaces place additional lexical and length requirements on compression systems. This study examines extractive sentence compression under  constraints, using both classical ILP-based methods and modern neural network techniques. We introduce a new neural compression method which can accommodate length and lexical requirements.
\end{abstract}

\section{Introduction}
Traditional study of extractive sentence compression seeks to create short, readable compressions which retain the most ``important'' information from source sentences. But query-focused and user-facing applications impose additional requirements on the output of a compression system. Compressions must be short enough to be shown in a user interface; and (often) compressions must contain a user's query term. An example of such a compression system is shown in figure \ahcomment{TODO}.

This study examines sentence compression under strict lexical and length constraints, in which a compression must be shorter than a given character budget and must include particular words. While older compression methods based on integer linear programming could trivially accommodate length and lexical restrictions, recent work has focused on neural network techniques which do not give practitioners such control. This makes existing neural methods unsuitable for search engines, concept map browsers and new forms of exploratory textual interfaces, where length and lexical constraints are paramount. \ahcomment{Cite hearst, IR book, falke, marchioni. examples.} Systems that can sometimes recreate gold training data are simply not useful for practitioners who must include query terms in shortened sentences; or scale compressions to a desktop interface or mobile phone.

\ahcomment{sign posting}

\section{Compression under constraints}

In this work, we present a new method for compressing sentences, which leverages the predictive power of neural methods while accommodating lexical and length constraints. We compare our method, \textbf{iterative deletion} to traditional \textbf{ILP-based techniques}, which also give a user such control.

\subsection{Constrained compression with ILPs}

One common approach for compressing sentences formulates compression as an integer linear programming (ILP) task. ILP-based methods assign binary variables to each token or nested subtree in a dependency parse, indicating if a word or subtree should be included in a compression. Each span, word or subtree in compression is also assigned a local weight, indicating its worthiness for inclusion in a shortened sentence.\footnote{Local weights are either learned via structured perceptron techniques; or inferred from corpus statistics, tf-idf scores or language models.}

ILP methods represent the overall quality of a compression by summing the local weights of sentence components to compute a global objective score. Researchers use off-the-shelf ILP solvers to identify the highest-scoring compression, from among the exponential possible configurations of binary variables. 

The integer linear programming approach easily accommodates constrained optimization. Often, researchers will add constraint terms to the ILP objective in order to preserve the meaning of a sentence (e.g. don't remove negations) or ensure that output forms a valid dependency tree (e.g. each subtree must have a parent vertex). Under such approaches, adding length or lexical requirements is trivial: practitioners must specify that optimal solutions must be shorter than some character budget and must specify that binary variables marking inclusion of particular words must be set to 1. 

\subsection{Iterative deletion}

\ahcomment{Details of what we implement}

\section{Outline}
\begin{enumerate}
\item{How do you do it? ILPs, neural net taggers, iterative deletion supervised w/ people or w/ gold data? (\S\ref{s:method}).}
\item{Why iterative deletion is good. Description of current system for making deletion decisions w/ human supervision. Possibly make this system better too}
\end{enumerate}

\section{Why iterative deletion is good}\label{s:method}
\begin{enumerate}
\item{Historical reasons}
\item{grammatical bias + syntax bias + linguistically-motivated? Leverages a lot that is know from linguistics: e.g. adjuncts, transitive bias, function words.}
\item{budget and query are trivial}
\item{linear (taggers) vs. quadratic (this work) vs. ILPs (exponential). }
\item{single op supervision can be used}
\item{avoid computational waste w/ grammar! HUGE divergence between mathematical properties of compression problem (exponential) and ways you can actually compress a sentence.}
\item{One thought: you could do a neural system that decides when/what to delete, using F and A as supervision? Each tree gets an encoding and you do a pointer network style softmax over trees? This sounds hard to implement.}
\end{enumerate}

\ahcomment{Here by nov 1 EOD}

\section{Outline}
\begin{enumerate}
\item{$(q,s,r)$ compression is useful.}
\item{How do you do it? ILPs, neural net taggers, iterative deletion supervised w/ people or w/ gold data? (\S\ref{s:method}).}
\item{Why iterative deletion is good. Description of current system for making deletion decisions w/ human supervision. Possibly make this system better too}
\item{Computational experiments (1): Properties of q,s,r compression in English. }
\item{Computational experiments (2): F and A data, interpreted as q,s,r supervision/data. Enumerate methods and bold the one with the best token-level F1 score.}
\item{Possible post hoc human experiment? Is our system actually good?}
\end{enumerate}



\subsection{Preprocessing with extract}
Extract preprocessing. Make up a few simple rules. They will help a bunch. Only like 6 dependency types work at all for extract. This is in realfake/createdata/extract at the moment.

\section{Computational experiments part 1: investigating properties of q,s,r compression}
\begin{enumerate}
\item{q = a list of 1 to 3 NER}
\item{r = random}
\item{What is the size of the minimum compression?}
\item{Reachability by budget by position of q in syntax tree. (If q is more than one entity then how the entities are dispersed across the tree probably matters a bunch too).}
\item{Hang on. reachability == min compression, eh? if min compression $>$ b, it is clearly bad.}
\item{Avoid computational waste w/ grammar.  Examine: ops you never have to worry about if you prune a branch v. dependency type deletion endorsement rate. Some ops get rid of lots of tokens w/ very high probability of deletion endorsements: e.g. parataxis (a great op!). By contrast: pruning a noun subj destroys acceptability and usually does not delete many tokens. Not worth the risk!}
\item{What is the empirical number of ops (i.e. decisions you have to make about pruning) if you greedily drop branches but never drop if the single op probability is less than $p$? My guess is you can make this problem way, way, simpler than implied by exponential formulation. Is it really quadratic?}
\item{Distribution of number of ops used for different q and r: when choosing ops at random? when choosing greedily? When pruning $\propto$ p(endorsement)?}
\item{Other stuff: min compression, reachability, operations saved w/ big prunes? position of query in the tree?}
\end{enumerate}

\section{Computational experiments part 2: sentence compression}
Interpret F and A as Q,S,R supervision/data. Then test methods which do $(q,s,r)$ compression. Measure w/ token level F1.


\begin{table}[htb!]
\begin{tabular}{ll}
\centering
Approach & F1 \\ \hline
Human acceptability supervision         &  A          \\
F and A (2013)    & B           \\
C and L (2008)    & C        \\
Neural subtree deletion methods? &  D    \\   
\end{tabular}
\end{table}

\begin{enumerate}
\item{q = a list of 1 to 3 NER which are in the compression. All the NER in the compression? If the compression has no NER, use noun unigrams I guess? or probably just skip it}
\item{r = length of gold}
\end{enumerate}

\ahcomment{what do they do in IR for query biased snippets? Hamed, John, Jieppu.
 (Prior study suggests that grammatical malformations cause readability errors in search results \cite{kanungo2009predicting}).}

\section{TODO}
\begin{enumerate}
\item{How will we deal with semantics? Brendan email about subsective adjectives ellie pavlik work. }
\end{enumerate}

\section{Related work}

Traditional study of sentence compression is closely aligned with text summarization techniques that select and  shorten sentences \cite{Knight2000StatisticsBasedS,vanderwende2007beyond,clarke2008global,Nenkova2012ASO}. 
In these settings, it is important for compressions to retain ``important'' information from source sentences because they must stand-in for longer sentences within summaries.

Our concern with lexical constraints is better suited to applications in which user queries define important information in documents. For instance, our length and lexically-constrained compressions could be used in information retrieval systems that summarize search results using query-based snippets on a search engine results page \cite{tombros1998advantages,Metzler2008MachineLS}. Often, such snippets must contain query terms. \ahcomment{screenshot?}

Our method could also be used for particular forms of query-focused summarization, such as summarizing people \cite{w04} or companies \cite{filippova2009company} which might require a hard lexical constraint. (Other work on query-focused summarization \cite{das} assumes softer constraints). 

Length and lexically constrained compressions could also be used as part of new forms of search user interfaces \cite{hearst2009search}, such as concept map browsers \cite{falke2017graphdocexplore}. Our interest in this problem arose from constructing one such novel search system.

\bibliography{abe}
\bibliographystyle{acl_natbib}

\end{document}
