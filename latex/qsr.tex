%
% File naacl2019.tex
%
%% Based on the style files for ACL 2018 and NAACL 2018, which were
%% Based on the style files for ACL-2015, with some improvements
%%  taken from the NAACL-2016 style
%% Based on the style files for ACL-2014, which were, in turn,
%% based on ACL-2013, ACL-2012, ACL-2011, ACL-2010, ACL-IJCNLP-2009,
%% EACL-2009, IJCNLP-2008...
%% Based on the style files for EACL 2006 by 
%%e.agirre@ehu.es or Sergi.Balari@uab.es
%% and that of ACL 08 by Joakim Nivre and Noah Smith

\documentclass[11pt,a4paper]{article}
\usepackage[hyperref]{naaclhlt2019}
\usepackage{times}
\usepackage{graphicx}
\usepackage{booktabs}
\usepackage[ruled,vlined]{algorithm2e}
\usepackage{latexsym}
\usepackage{amsmath}
\usepackage{amssymb}
\usepackage{bm}

\newcommand{\rdep}[1]{\ $\xrightarrow{\text{\tiny #1}}$\ }
\newcommand{\speedup}[0]{4x~}
\newcommand{\exact}[0]{4.2~}
\newcommand{\ahcomment}[1]{\textcolor{blue}{[#1 -AH]}}
\newcommand{\ilptest}[0]{0.76~}

\usepackage{url}

\aclfinalcopy % Uncomment this line for the final submission
%\def\aclpaperid{***} %  Enter the acl Paper ID here

%\setlength\titlebox{5cm}
% You can expand the titlebox if you need extra space
% to show all the authors. Please do not make the titlebox
% smaller than 5cm (the original size); we will check this
% in the camera-ready version and ask you to change it back.

%\hypersetup{draft} % THIS IS NEEDED TO GET IT TO COMPILE. Does not like the tables. AH 11/27

\newcommand\BibTeX{B{\sc ib}\TeX}

\DeclareMathOperator*{\argmaxA}{arg\,max} % Jan Hlavacek

% Linear-time Sentence Compression under Lexical and Length Constraints
% I think we should call our method query-focused because it sells why you want this in the first place. We are similar enuf to query-focused summairzation I think ths is fine.
\title{Query-focused Sentence Compression in Linear Time}

\author{Abram Handler {\normalfont and} Brendan O'Connor \\
  College of Information and Computer Sciences \\
  University of Massachusetts, Amherst \\
  {\tt ahandler@cs.umass.edu}  \\}

\date{}

\begin{document}
\maketitle

\begin{abstract}
Search applications often display shortened sentences which must contain certain query terms and must fit within the space constraints of a user interface. This work introduces a new transition-based sentence compression technique developed for such settings. Our method constructs length and lexically constrained compressions in linear time, by growing a forest in the dependency parse of a sentence. This approach achieves a \speedup speed up over baseline ILP compression techniques, and better reconstructs gold shortenings under constraints. Such efficiency gains permit constrained compression of multiple sentences, without unreasonable lag.
\end{abstract}


\section{Introduction}\label{s:intro}

Traditional study of extractive sentence compression seeks to create short, readable, single-sentence summaries which retain the most ``important'' information from source sentences. But search applications often require compressions which must include a user's query terms and must not exceed some maximum length, permitted by screen space.  Figure \ref{f:qf} shows an example.

This study examines the English-language compression problem with such length and lexical requirements. In our constrained compression setting, a source sentence $S$ is shortened to a compression $C$ which (1) must include all tokens in a set of query terms $Q$ and (2) must be no longer than a maximum budgeted character length, $b \in \mathbb{Z}^{+}$. Formally, constrained compression maps $(S,Q,b) \rightarrow C$, such that $C$ respects $Q$ and $b$. We describe this task as query-focused compression because $Q$ places a hard requirement on words from $S$ which must be included in $C$.

\begin{figure}[htb!]
\includegraphics[width=8.5cm]{qf.pdf}
\caption{A search user interface (boxed, top) returns a snippet consisting of three compressions which must contain a users' query $Q$ (bold) and must not exceed $b=$ 75 characters in length. The third compression $C$ was derived from source sentence $S$ (italics, bottom).}
\label{f:qf}
\end{figure}

Existing techniques are poorly suited to constrained compression. While methods based on integer linear programming (ILP) can trivially accommodate such length and lexical restrictions \cite{clarke2008global,filippova2013overcoming,Wang2017CanSH}, these approaches rely on slow third-party solvers to optimize an NP-hard integer linear programming objective, causing interface lag. An alternative LSTM tagging approach \cite{filippova2015sentence} does not allow practitioners to specify length or lexical constraints, and requires an expensive graphics processing unit (GPU) to achieve low runtime latency (a barrier in fields like social science and journalism). These deficits prevent application of existing compression techniques in search user interfaces \cite{marchionini2006exploratory,hearst2009search}, where length, lexical and latency requirements are paramount. We thus present a new stateful method for query-focused compression. 

Our approach is theoretically and empirically faster than ILP-based techniques, and more accurately reconstructs gold standard compressions.


\begin{table}[htb!]
\begin{tabular}{lcc}
Approach & Complexity & Constrained  \\ \hline
\textsc{ilp}       &   exponential    & yes     \\
LSTM tagger & linear              & no         \\   
\textbf{\textsc{vertex addition}} & \textbf{linear}     &      \textbf{yes}   
\end{tabular}
\caption{Our \textsc{vertex addition} technique (\S\ref{s:system}) constructs constrained compressions in linear time. Prior work (\S\ref{s:relatedwork}) has higher computational complexity (\textsc{ILP}) or does not respect hard constraints (LSTM tagger).} 
\label{t:algos}
\end{table}

\section{Related work}\label{s:relatedwork}

Extractive compression shortens a sentence by removing tokens, typically for summarization \cite{Knight2000StatisticsBasedS,clarke2008global,filippova2015sentence,Wang2017CanSH}.\footnote{Some methods compress via generation instead of deletion \cite{rush2015neural,mallinson18}. Our extractive method is intended for practical, interpretable and trustworthy search systems \cite{Chuang2012InterpretationAT}. Users might not trust abstractive summaries \cite{Zhang:2018:MSG:3290265.3274465}, particularly in cases with semantic error.} To our knowledge, this work is the first to consider extractive compression under hard length and lexical constraints.

We compare our \textsc{vertex addition} approach to ILP-based compression methods \cite{clarke2008global,filippova2013overcoming,Wang2017CanSH}, which shorten sentences using an integer linear programming objective. \textsc{ilp} methods can easily accommodate lexical and budget restrictions via additional optimization constraints, but require worst-case exponential computation.\footnote{ILPs are exponential in $|V|$ when selecting \cite{clarke2008global} tokens and exponential in $|E|$ when selecting edges \cite{filippova2015sentence}.} 

Finally, compression methods based on LSTM taggers \cite{filippova2015sentence} cannot currently enforce lexical or length requirements. Future work might address this limitation by applying and modifying constrained generation techniques \cite{D16-1140,N18-1119,D18-1443}.

\section{Compression via \textsc{vertex addition}}\label{s:system}

We present a new transition-based method for shortening sentences under lexical and length constraints, inspired by similar approaches in transition-based parsing \cite{nivre2003}. We describe our technique as \textsc{vertex addition} because it constructs a shortening by \textit{growing} a forest in the dependency parse of a sentence, one vertex at a time. This approach can construct constrained compressions with a linear algorithm, leading to \speedup lower latency than ILP techniques (\S\ref{s:autoeval}). To our knowledge, our method is also the first to construct compressions by $\textit{adding}$ vertexes by growing a forest, rather than \textit{pruning} subtrees in a parse \cite{Knight2000StatisticsBasedS,berg2011jointly,almeida2013fast,Filippova2015FastKS}. Note that this approach assumes a boolean relevance model: $S$ must contain $Q$. 

\subsection{Formal description}\label{s:formal}

\textsc{vertex addition} builds a compression by maintaining a state
$(C_i,P_i)$ where $C_i \subseteq S$ is a set of added candidates, $P_i  \subseteq S$ is a priority queue of vertexes, and $i$ indexes a timestep during compression. Figure \ref{f:walkthru} shows a step-by-step example. 

During initialization, we set $C_0 \gets Q$ and $P_0 \gets S \setminus Q$. Then, at each timestep, we pop some candidate $v_i =h(P_0)$ from the head of $P_0$ and evaluate $v_i$ for inclusion in $C_i$. (Neighbors of $C_i$ in $P_i$ get higher priority than non-neighbors; we break ties in left-to-right order, by sentence position.) If we accept $v_i$, then $C_i \gets C_i \cup v_i$. We discuss acceptance decisions in detail in \S\ref{s:transition}. We continue adding vertexes to $C$ until either $P_i$ is empty or $C_i$ is $b$ characters long.\footnote{We linearize $C$ by left-to-right vertex position in $S$, common for compression in English \cite{filippova2013overcoming}.} The appendix includes a formal algorithm. 

\textsc{vertex addition} is linear in the token length of $S$ because we pop and evaluate some vertex from $P_i$ at each timestep, after $P_0  \gets S \setminus Q$. Additionally, because (1) we never accept $v_i$ if the length of $C_i \cup v_i$ is more than $b$, and (2) we set $C_0 \gets Q$, our method respects $Q$ and $b$.

\begin{figure}[h]
\includegraphics[width=8.2cm]{additive.pdf}
\caption{A dependency parse of a sentence $S$, shown across five timesteps of \textsc{vertex addition} (from left to right). Each node in the parse is a vertex in $S$. Our stateful method produces the final compression $\{$A,C,B,E$\}$ (rightmost). At each timestep, each candidate $v_i$ is boxed; rejected candidates $\neg C_i$ are unshaded.}
\label{f:walkthru}
\end{figure}

\section{Evaluation}\label{s:autoeval}

We observe the latency, readability and token-level F1 score of \textsc{vertex addition}, using a standard dataset (\S\ref{s:constrained}).
We compare our method to an \textsc{ilp} baseline (\S\ref{s:relatedwork}) because ILP methods are the only known technique for constrained compression. All methods have similar compression ratios (shown in appendix), a well-known evaluation requirement \cite{napoles2011evaluating}. 
\textsc{vertex addition} achieves higher F1 and slightly lower readability scores (Table \ref{t:results}), with \speedup lower latency  than the \textsc{ilp}. We evaluate the significance of differences between \textsc{vertex addition} and the \textsc{ilp} with bootstrap sampling \cite{D12-1091}. All differences are significant {\small $(p < .01)$}. 

\subsection{Constrained compression experiment}\label{s:constrained}

In order to evaluate different approaches to constrained compression, we require a dataset of sentences, constraints and known-good shortenings, which respect the constraints. This means we need tuples $(S, Q, b, C_g)$, where $C_g$ is a known-good compression of $S$ which respects $Q$ and $b$ (\S\ref{s:intro}).

To support large-scale automatic evaluation, we reinterpret a standard compression corpus \cite{filippova2013overcoming}
as a collection of input triples and constrained compressions. The original dataset contains pairs of sentences $S$ and compressions $C_g$, generated using headlines. For our experiment, we set $b$ equal to the character length of the gold compression $C_g$. We then sample some query set $Q$ from the proper and improper nouns in $C_g$ so that the distribution of cardinalities of queries across our dataset simulates the observed distribution of cardinalities (i.e.\ number of query tokens) in real-world search \cite{Jansen2000RealLR}. Sampled queries are short sets of nouns, such as ``police, Syracuse'', ``NHS'' and ``Hughes, manager, QPR,'' approximating real-world behavior \cite{Barr2008TheLS}.\footnote{See appendix for detailed discussion of query sampling.} 

By sampling queries and defining budgets in this manner, we create {199,152} training tuples and {9,969} test tuples, each of the form $(S,Q,b,C_g)$. \citet{filippova2013overcoming} define the train/test split. We re-tokenize, parse and tag with CoreNLP v3.8.0 \cite{corenlp}. We reserve 24,999 training tuples as a validation set. 

\subsection{Model: \textsc{ilp}}\label{s:ilp}

We compare our system to a baseline, \textsc{ilp} method, presented in \citet{filippova2013overcoming}. This approach represents each edge in a syntax tree with a vector of real-valued features, then learns feature weights using a structured perceptron trained on a corpus of $(S,C_g)$ pairs.\footnote{Another ILP \cite{Wang2017CanSH} sets weights using a LSTM, achieving similar in-domain performance. This method requires a multi-stage computational process (i.e.\ run LSTM \textit{then} ILP) that is poorly-suited to our query-focused setting, where low latency is crucial.} Learned weights are used to compute a global compression objective, subject to structural constraints which ensure $C$ is a valid tree. This baseline can easily perform constrained compression: at test time, we add optimization constraints specifying that $C$ must include $Q$, and not exceed length $b$.

To our knowledge, a public implementation of this method does not exist. We reimplement from scratch using \citet{gurobi}, achieving a test-time, token-level F1 score of \ilptest on the unconstrained compression task, lower than the result {\small (F1 = 84.3)} reported by the original authors. There are some important differences between our reimplementation and original approach (described in detail in the appendix). Note that \textsc{vertex addition} requires $Q$ and $b$, so we can only compare it to the ILP on the \textit{constrained} (rather than traditional, unconstrained) compression task.

\subsection{Models: \textsc{vertex addition}}\label{s:transition}

Vertex addition accepts or rejects some candidate vertex $v_i$ at each timestep $i$. 
We learn such decisions $y_i \in \{0,1\}$ using a corpus of tuples $(S,Q,b,C_g)$ (\S\ref{s:constrained}). Given such a tuple, we can always execute an oracle path shortening $S$ to $C_g$ by first initializing \textsc{vertex addition} and then, at each timestep: (1) choosing $v_i = h(P_i)$ and (2) adding $v_i$ to $C_i$ iff $v_i \in C_g$. We say that each $y_i=1$ if $v_i \in C_g$, and that $y_i=0$ if $v_i \notin C_g$. We use decisions from oracle paths to train two models of inclusion decisions, ${p(y_i  = 1 | v_i, C_i, P_i, S)}$.

The model {\textsc{vertex addition}$_{NN}$ broadly follows the \citet{D14-1082} approach to transition-based parsing: we predict $y_i$ using a LSTM classifier with a standard max-pooling architecture \cite{D17-1070}, implemented with a common neural framework \cite{Gardner2017AllenNLP}. Our classifier maintains four vocabulary embeddings matrixes, corresponding to the four disjoint subsets $C \cup \neg C \cup P \cup \{v_i\} = V$. Each LSTM input vector $x_t$ comes from the appropriate embedding for $v_t \in V$, depending on the state of the compression system at timestep $i$. The appendix details hyperparameter tuning and network optimization.

The model \textsc{vertex addition}$_{LR}$ uses binary logistic regression,\footnote{We implement with Python 3 using scikit-learn \cite{Pedregosa:2011:SML:1953048.2078195}. We tune the inverse regularization constant to $c=10$ via grid search over powers of ten, to optimize validation set F1.} with features that fall into 3 classes.

\textbf{Edge features} describe the properties of the edge $(u,v_i)$ between $v_i \in P_i$ and $u \in C_i$. We use the edge-based feature function from \citet{filippova2013overcoming}, described in detail in the appendix. This allows us to compare the performance of a vertex addition method based on local decisions with an ILP method that optimizes a global objective (\S \ref{s:ablated}), using the same feature set.

\textbf{Stateful features} represent the relationship between $v_i$ and the compression $C_i$ at timestep $i$. Stateful features include information such as the position of $v_i$ in the sentence, relative to the right-most and left-most vertex in $C_i$, as well as history-based information such as the fraction of the character budget used so far. Such features allow the model to reason about which sort of $v_i$ should be added, given $Q$, $S$ and $C_i$.

\textbf{Interaction features} are formed by crossing all stateful features with the type of the dependency edge governing $v_i$, as well as with indicators identifying if $u$ governs $v_i$, if $v_i$ governs $u$ or if there is no edge $(u,v_i)$ in the parse.

\subsection{Analysis:  \textsc{ablated} \& \textsc{random}}\label{s:ablated}
For comparison, we implement an \textsc{ablated} vertex addition method, which makes inclusion decisions using a model trained on only edge features from \citet{filippova2013overcoming}. This method achieves a lower F1 than the ILP, which integrates the same edge-level information to optimize a global objective. However, adding stateful and interaction features (Table \ref{t:results}) to inform local vertex addition decisions raises F1 score. The relatively strong performance of \textsc{ablated} hints that edge-level information alone can largely guide $v_i$ decisions, e.g.\ should some verb $v_i$ governing some object in $C_i$ via $(v_i,u)$ be included?

We also evaluate a \textsc{random} baseline, which accepts each $v_i$ randomly in proportion to $p(y_i = 1)$ across training data. \textsc{random} achieves reasonable F1 because (1) $C_0 = Q \in C_g$ and (2) F1 correlates with compression rate \cite{napoles2011evaluating}, and $b$ is set to the length of $C_g$.

\subsection{Importance and readability evaluation}\label{s:readabilityinformativeness}

Researchers often use human judgements of \textit{importance} and \textit{readability} to evaluate compression techniques \cite{Knight2000StatisticsBasedS,filippova2015sentence}. In our setting $Q$ determines the ``important'' information from $S$, so human importance evaluations are inappropriate.

We use the automated SLOR metric \cite{lau2015unsupervised} to check the readability of compressions. SLOR is known to correlate with human readability judgements \cite{kannConl}. The appendix details our implementation of SLOR.

\subsection{Latency evaluation}\label{s:costs}

\textsc{vertex addition} has lower theoretical complexity than the \textsc{ilp} baseline. We test corresponding latency gains by measuring the speed of each compression method, in milliseconds per sentence. The appendix describes measurement details. \textsc{vertex addition}$_{LR}$ is on average \speedup faster than the \textsc{ilp} (Table \ref{t:results}). Added speedups might be possible by reimplementing our Python 3 code in a more performant language such as C. We measure latency of \textsc{vertex addition}$_{NN}$ using a CPU. This method is impractical in search applications that do not use a GPU.

\begin{table}[]
\begin{tabular}{lccc}
\centering
Approach & F1 & SLOR & $^{*}$Latency \\ \hline
\textsc{random} {\small (lower bound) }&{\small 0.653}&{\small 0.371}& {\small 0.5}\\
\textsc{ablated} {\small (edge only) }&{\small 0.820}&{\small 0.665}&{\small 3.5}\\
\textsc{vertex addition}$_{NN}$& {\small 0.884}& {\small 0.731}& {\small 98.4} \\ \midrule 
\textsc{ilp}&{\small 0.856}& {\small 0.755}&{\small 21.9}\\
\textsc{vertex addition}$_{LR}$ & \small 0.881 & {\small 0.745}& \small 5.2 \\
\end{tabular}
\caption{Test results for constrained compression. $^*$Latency is shown in milliseconds (ms.) per sentence. \textsc{vertex addition}$_{LR}$ achieves higher F1 than the \textsc{ilp}. The method is also \exact times faster. Differences between all scores for \textsc{vertex addition}$_{LR}$ and \textsc{ilp} are significant {\tiny $(p < .01)$}. {\textsc{vertex addition}$_{NN}$} achieves the highest F1, but runs slowly on a CPU.}
\label{t:results}
\end{table}

\section{Conclusion}

Both novel and traditional search user interfaces would benefit from low-latency, query-focused  compression. We introduce a new \textsc{vertex addition} technique for such settings, which is \speedup faster than an \textsc{ilp} baseline while achieving higher F1 scores. 

In search applications, the latency gains from \textsc{vertex addition}$_{LR}$ are non-trivial: real users are measurably hindered by interface lags \cite{Nielsen,Liu2014TheEO}. Fast, query-focused compression better enables search interfaces to generate query-focused snippets at the ``pace of human thought'' \cite{heerschei}.


%Finally, some shortened sentences will modify the meaning of a sentence. Identifying such cases is a special case of the unsolved textual entailment problem \cite{snli_bowman,Pavlick2016SoCalledNA,linzencompression}. In the future, we plan to apply entailment research to the compression task.  

% ack => Katie, Javier, Nick Eubank, NLP reading group! 

%\appendix

\section{Appendix}

\subsection{Algorithm}
We formally present the \textsc{vertex addition} compression algorithm, using notation defined in the paper. $\ell$ linearizes a vertex set, based on left-to-right position in $S$; $|P|$ indicates the number of tokens in the priority queue.

\begin{algorithm}[]
\SetKwInOut{Input}{input}
\SetAlgoLined
\Input{$s=(V,E)$, $Q \subseteq V$, $b \in \mathbb{R^+}$}
 $ C \gets Q;  P \gets V \setminus Q$; \\
 \While{ $\ell(C) < b $ and $ |P| > 0 $   }{
  $v \gets \text{pop}(P)$; \\
  \If{$p(y=1) > .5$ and $\ell(C \cup \{v\}) \leq b$}{$C \gets C \cup \{v\}$}
 }
\KwRet{$\ell(C)$}
 \caption{\textsc{vertex addition}}
\end{algorithm}\label{a:algo}

\subsection{Reimplementation of \citet{filippova2013overcoming}}

In this work, we reimplement the method of \citet{filippova2013overcoming}, who in turn implement a method partially described in \citet{filippova2008dependency}.  There are inevitable discrepancies between our implementation and the methods described in these two prior papers.  

Some discrepancies arise from differences in syntactic formalisms, following from our choice to implement using the well-known Universal Dependencies (UD) v1 framework \cite{Nivre2016UniversalDV}. Most importantly, prior work implements a tree transformation as a preprocessing step. Some steps of this transform are incompatible with UD,\footnote{e.g.\ the tree transformation from prior work assumes PPs are headed by prepositions, which is not true in UD} and other steps are no longer necessary given changes in dependency representation. We make the following modifications.

In prior work, \citet{filippova2013overcoming} augment edge labels in the original dependency parse during transformation. We implement this step by assigning edge labels using the enhanced dependencies representation in CoreNLP \cite{Schuster2016EnhancedEU}. The augmented modifiers and augmented conjuncts from enhanced dependencies are very similar to the augmented edge labels from prior work.

\citet{filippova2013overcoming} also transform trees by adding an edge between the root node and all verbs in a sentence. (This step ensures that subclauses can be removed from parse trees, and then merged together to create a compression from different clauses of a sentence.) During development, we found that replicating this transform literally (i.e. only adding edges from the original root to all tokens tagged as verbs) made it impossible for the ILP to recreate some of the gold compressions in the dataset. (We suspect that this is because our part-of-speech tagger and the original part-of-speech tagger employed in \citet{filippova2013overcoming} sometimes return different part-of-speech tags.) Our tree transform therefore adds an edge between the root node and \textit{all} tokens in a sentence. With this change, it is always possible for the ILP to return the gold compression.

Other differences in implementation are not related to the tree transform.

\begin{itemize}
\item{We use \citet{gurobi} (v8) to solve the liner program. \citet{filippova2008dependency} report using LPsolve.\footnote{\url{http://
sourceforge.net/projects/lpsolve}}}
\item{We exclude a syntactic constraint based on the \rdep{sc} relation, which is not included in UD.}
\item{In Table 1 of their original paper, \citet{filippova2013overcoming} provide an overview of the syntactic, structural, semantic and lexical features in their model. We implement every feature explicitly described in their work, except where otherwise noted (e.g. syntactic feature not compatible with UD). However, additional features included in their model (but not explicily described in print) almost certainly affect performance.}
\end{itemize}

Finally, we assess convergence of the ILP by examining the validation F1 score on the traditional sentence compression task after each pass through the training data. We observe that F1 increases for each of five passes through the training data, and then decreases slightly (drops by $10^{-3}$ points). We thus terminate training after five epochs. Where the original authors train on only 100,000 sentences, we use the full training set to enable are more fair comparison with additive compression, which trains on all available data.

\subsection{Implementation of SLOR}

We use the SLOR function to measure the readability of the shortened sentences produced by each compression system. Following \cite{lau2015unsupervised}, we define the function as 

\begin{equation}
\text{SLOR}=\frac{\text{log}P_m(\xi) - \text{log}P_u(\xi)}{|\xi|}
\end{equation}

where $\xi$ is a sequence of words, $P_u(\xi)$ is the unigram probability of this sequence of words and $P_m(\xi)$ is the probability of the sequence, assigned by a language model.  $|\xi|$ is the length (in tokens) of the sentence.

We use a 3-gram language model trained on the \citet{filippova2013overcoming} corpus. We implement with KenLM \cite{Heafield-kenlm}.

\subsection{Latency evaluation}
To measure latency, for each technique, we sample 100,000 sentences with replacement from the test test. We observe the mean time per sentence using Python's built in \textit{timeit} module. Code for extracting edge features is shared across both methods, to ensure that each technique incurs very similar latency penalties for edge feature extraction. 

\subsection{Query sampling}
We sample queries for our synthetic constrained compression experiment to mimic real-world searches: the distribution of query token lengths and the distribution of query part-of-speech tags employed in our experiment closely match empirical distributions observed in real search \cite{Jansen2000RealLR,Barr2008TheLS}. To create queries, for each sentence in our corpus we: (1) sample a query token length in proportion to the real-world distribution over query token lengths (2) sample a proper or common noun in proportion to the real-world distribution over proper and improper nouns in queries, (3) add a noun or proper noun from the gold compression $C_g$ (based on step 2) to the set $\{Q\}$, if such a token exists (4) repeat steps 2 and 3 until $\{Q\}$ is the length specified in step 1. We exclude sentences in cases where the gold compression $C_g$ does not contain enough nouns to fill $\{Q\}$ to the desired length.

\bibliography{abe}
\bibliographystyle{acl_natbib}

\bibliography{abe}
\bibliographystyle{acl_natbib}

\end{document}