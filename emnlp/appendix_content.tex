\appendix

 \ahcomment{Discuss tuning appendix}

\section{Appendix}

We present the \textsc{vertex addition} compression technique, using notation defined in the paper. $\ell$ linearizes a vertex set, based on left-to-right position in $S$; $|P|$ indicates the number of tokens in the priority queue.

\begin{algorithm}[]
\SetKwInOut{Input}{input}
\SetAlgoLined
\Input{$s=(V,E)$, $Q \subseteq V$, $b \in \mathbb{R^+}$}
 $ C \gets Q;  P \gets V \setminus Q$; \\
 \While{ $\ell(C) < b $ and $ |P| > 0 $   }{
  $v \gets \text{pop}(P)$; \\
  \If{$p(y=1) > .5$ and $\ell(C \cup \{v\}) \leq b$}{$C \gets C \cup \{v\}$}
 }
\KwRet{$\ell(C)$}
 \caption{\textsc{vertex addition}}
\end{algorithm}\label{a:algo}


\subsection{Query sampling}
We sample queries for our synthetic constrained compression experiment to mimic real-world searches: the distribution of query token lengths and the distribution of query part-of-speech tags employed in our experiment closely match empirical distributions observed in real search \cite{Jansen2000RealLR,Barr2008TheLS}. To create queries, for each sentence in our corpus we: (1) sample a query token length in proportion to the real-world distribution over query token lengths, (2) sample a part-of-speech tag in proportion to the real-world distribution over part-of-speech tags, (3) add a token from the sentence with the sampled tag to the set $\{Q\}$, if such a token exists (4) repeat steps 2 and 3 until $\{Q\}$ is the length specified in step 1.
 
We reject samples in which the query length is longer than 6 tokens, which are poorly suited our compression setting. We coarsen Penn tags to match parts of speech described in prior work and we reject sampled part-of-speech tags which occur in fewer than 5\% of real queries.

\subsection{Reimplementation of \citet{filippova2013overcoming}}

In this work, we reimplement the method of \citet{filippova2013overcoming}, who in turn implement a method partially described in \citet{filippova2008dependency}.  There are inevitable discrepancies between our implementation and the methods described in these two prior papers.  

Some discrepancies arise from differences in syntactic formalisms, which arise from our choice to implement using Universal Depenencies v1 \cite{Nivre2016UniversalDV}. For instance, prior work uses a tree transformation method which is no longer strictly compatible with UD (e.g.~the tree transformation from prior work assumes PPs are headed by prepositions, which is not true \cite{Schuster2016EnhancedEU} in UD). We thus reimplement the tree transformation, using the enhanced dependencies representation from CoreNLP, which provides off-the-shelf augmented modifiers and augmented conjuncts that are very similar to the augmented edge labels from prior work. We exclude a syntactic constraint based on the \rdep{sc} relation, which is not included in UD.

Other possible discrepancies arise from differences in part-of-speech taggers. In particular, the aforementioned tree transform from prior work adds an edge between the root node and all verbs in a sentence, as a preprocessing step. This ensures that subclauses can be removed from parse trees, and then merged together to create a compression from different clauses of a sentence. However, we found that replicating this transform literally (i.e. only adding edges from the original root to all ``verbs'') made it impossible for the ILP to recreate some of the gold compressions in the dataset. (We suspect that this is because our part-of-speech tagger and the original part-of-speech tagger employed in \citet{filippova2013overcoming} sometimes return different part-of-speech tags). Our tree transform therefore adds an edge between the root node and \textit{all} tokens in a sentence. With this change, it is always possible for the ILP to output the gold compression.

We use \citet{gurobi} (v8) to solve the liner program. \citet{filippova2008dependency} report using LPsolve.\footnote{\url{http://
sourceforge.net/projects/lpsolve}}  We assess convergence by examining the validation F1 score on the unconstrained task after each pass through the training data. The F1 score increases for each of five passes through the training data, and then decreases slightly (drops by $10^{-3}$ points). We terminate training at this point. Where the original authors only used 100,000 training instances, we use the full training set (175,000 instances) to enable are more fair comparison with additive compression, which trains on the full set.

Lastly, in Table 1 of their original paper, \citet{filippova2013overcoming} provide an overview of the syntactic, structural, semantic and lexical features in their model. We implement every feature explicitly described in their work, except where otherwise noted (e.g. syntactic feature not compatible with UD). However, additional features included in their model (but not explicily described in print) almost certainly affect performance. 

\subsection{Implementation of SLOR}

We use the SLOR function to measure the readability of the shortened sentences produced by each compression system. Following \cite{lau2015unsupervised}, we define the function as 

\begin{equation}
\text{SLOR}=\frac{\text{log}P_m(\xi) - \text{log}P_u(\xi)}{|\xi|}
\end{equation}

where $\xi$ is a sequence of words, $P_u(\xi)$ is the unigram probability of this sequence of words and $P_m(\xi)$ is the probability of the sequence, assigned by a language model.  $|\xi|$ is the length (in tokens) of the sentence.

We use a 3-gram language model trained on the \citet{filippova2013overcoming} corpus. We implement with KenLM \cite{Heafield-kenlm}.

\subsection{Latency evaluation}
To measure latency, for each technique, we sample 100,000 sentences with replacement from the test test. We observe the mean time per sentence usings Python's built in \textit{timeit} module.

\bibliography{abe}
\bibliographystyle{acl_natbib}