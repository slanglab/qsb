%
% File naacl2019.tex
%
%% Based on the style files for ACL 2018 and NAACL 2018, which were
%% Based on the style files for ACL-2015, with some improvements
%%  taken from the NAACL-2016 style
%% Based on the style files for ACL-2014, which were, in turn,
%% based on ACL-2013, ACL-2012, ACL-2011, ACL-2010, ACL-IJCNLP-2009,
%% EACL-2009, IJCNLP-2008...
%% Based on the style files for EACL 2006 by 
%%e.agirre@ehu.es or Sergi.Balari@uab.es
%% and that of ACL 08 by Joakim Nivre and Noah Smith

\documentclass[11pt,a4paper]{article}
\usepackage[hyperref]{naaclhlt2019}
\usepackage{times}
\usepackage{graphicx}
\usepackage{booktabs}
\usepackage[ruled,vlined]{algorithm2e}
\usepackage{latexsym}
\usepackage{amsmath}
\usepackage{amssymb}
\usepackage{bm}

\newcommand{\rdep}[1]{\ $\xrightarrow{\text{\tiny #1}}$\ }
\newcommand{\speedup}[0]{100X~}
\newcommand{\ahcomment}[1]{\textcolor{blue}{[#1 -AH]}}

\usepackage{url}

%\aclfinalcopy % Uncomment this line for the final submission
%\def\aclpaperid{***} %  Enter the acl Paper ID here

%\setlength\titlebox{5cm}
% You can expand the titlebox if you need extra space
% to show all the authors. Please do not make the titlebox
% smaller than 5cm (the original size); we will check this
% in the camera-ready version and ask you to change it back.

%\hypersetup{draft} % THIS IS NEEDED TO GET IT TO COMPILE. Does not like the tables. AH 11/27

\newcommand\BibTeX{B{\sc ib}\TeX}

\DeclareMathOperator*{\argmaxA}{arg\,max} % Jan Hlavacek

% Extractive sentence compression under lexical and length constraints
\title{Additive Compression with Lexical and Length Constraints}

\author{First Author \\
  Affiliation / Address line 1 \\
  Affiliation / Address line 2 \\
  Affiliation / Address line 3 \\
  {\tt email@domain} \\\And
  Second Author \\
  Affiliation / Address line 1 \\
  Affiliation / Address line 2 \\
  Affiliation / Address line 3 \\
  {\tt email@domain} \\}

\date{}

\begin{document}
\maketitle

\begin{abstract}
Extractive sentence compression seeks to reduce the length of a sentence, while retaining the most ``important'' information from the source. But query-focused applications such as document search engines or exploratory search interfaces place additional length, lexical and latency constraints on compression systems. This study introduces a new transition-based technique, developed for such settings. Our method constructs compressions in linear time by growing a forest in a dependency parse, achieving a \speedup speed up over traditional ILP-based methods. Our technique also better reconstructs known-good shortenings under constraints.
\end{abstract}

\section{Introduction}\label{s:intro}

Traditional study of extractive sentence compression seeks to create short, readable compressions which retain the most ``important'' information from source sentences. But query-focused and user-facing applications impose additional requirements on the output of a compression system. Compressions must be short enough to be shown in a user interface and often must contain a user's query term, which defines the ``important'' information in the sentence. Query-focused applications applications also require low latency to support interactive browsing. An example application is shown in Figure \ref{f:qf}.

\begin{figure}[htb!]
%\fbox{
%\begin{minipage}{7.6 cm}
\includegraphics[width=8.5cm]{qf.pdf}
\caption{A search interface (boxed, top) returns a headline $h$ above a constrained compression $C$, shortened from a sentence $S$ in a query-relevant document (italics, bottom). The constrained compression must contain $Q$, the users' query terms (bold), and must not exceed $b=$75 characters in length.}
\label{f:qf}
%\end{minipage}
%}
\end{figure}

This study examines the English-language compression problem with such length and lexical requirements. In constrained compression, source sentences $S$ are shortened to compressions $C$ which (1) must include query words $Q$ and (2) must be shorter than or equal in length to a maximum character length, $b \in \mathbb{Z}^{+}$. Formally, the task is to construct a function $(S,Q,b) \rightarrow C$, such that $C$ respects $Q$ and $b$.

Older compression methods  \cite{clarke2008global,filippova2013overcoming} based on integer linear programming can trivially accommodate such restrictions, but rely on slow third-party solvers to optimize an NP-hard integer linear programming objective\label{s:relatedwork}. Newer sequence to sequence methods \cite{filippova2015sentence} do not allow practitioners to specify length or lexical constraints and require access to specialized graphics processing hardware (GPU) to achieve low latency, presenting a barrier for practitioners in fields like social science and journalism. These deficits make existing compression techniques unsuitable for search engines \cite{hearst2009search}, concept map browsers \cite{falke2017graphdocexplore} and new forms of exploratory textual interfaces \cite{marchionini2006exploratory}, where length, lexical and latency constraints are paramount. 

\begin{table*}[htb!]
\begin{tabular}{lccc}
\textbf{Approach} & \textbf{Worst-case complexity} & \textbf{Constrained}  \\ \hline
ILP    \cite{filippova2013overcoming,Wang2017CanSH}       &   exponential    & yes     \\
Sequence to sequence taggers \cite{filippova2015sentence}   & linear              & no         \\   
\textbf{Additive compression (this work)}  & \textbf{linear}     &      \textbf{yes}   
\end{tabular}
\caption{Supervised ILP compression methods \cite{clarke2008global,filippova2013overcoming,Wang2017CanSH} can easily accommodate length and lexical restrictions. But these methods must solve a known NP-hard problem with worst-case exponential runtime. LSTM taggers \cite{filippova2015sentence} achieve comparable results with linear runtime, but cannot accommodate length or lexical requirements. This work introduces a supervised additive approach (\S\ref{s:system}) which constructs constained compressions in linear time.} 
\label{t:algos}
\end{table*}

Thus, in this work, we present a new method for constrained compression which grows a forest in a dependency parse in linear time. We compare this \textit{additive compression} technique to supervised, integer linear programming (ILP) systems, which also accommodate length and lexical restrictions. Our method has lower theoretical and empirical computational costs, while better reconstructing known-good shortenings. 

\section{Related work}\label{s:relatedwork}

Extractive compression \cite{Knight2000StatisticsBasedS,clarke2008global,filippova2015sentence,Wang2017CanSH} shortens a sentence by removing tokens, often to aid in a summarization task \cite{Knight2000StatisticsBasedS,almeida2013fast,P16-1188}.\footnote{Some methods shorten sentences via generation instead of deletion \cite{rush2015neural,mallinson18}. Our interest in the extractive setting follows from a motivation to create interpretable,  trustworthy, and practical search systems \cite{Chuang2012InterpretationAT}: users might not trust abstractive summaries \cite{Zhang:2018:MSG:3290265.3274465}, particularly in cases with semantic error.} To our knowledge, this work is the first to consider extractive compression under length and lexical constraints.\footnote{\citet{Li2013DocumentSV} solicit annotations for ``guided'' compression, but do not examine the compression problem under lexical and length constraints.}

Our interest in the constrained compression problem is motivated by search user interfaces \cite{hearst2009search}, which often display query-biased \cite{tombros1998advantages} and length-restricted shortenings.\footnote{Apache Lucene {\small (v7.7)} does not compress sentences to form snippets \cite{lucene}} 
%\url{https://lucene.apache.org/core/7_5_0/highlighter/index.html}} 
Because such user-facing systems require low-latency \cite{Nielsen,heerschei,Liu2014TheEO}, we adopt a computationally efficient approach. 

We compare our compression method to ILP-based compression techniques \cite{clarke2008global,filippova2008dependency,filippova2013overcoming,Wang2017CanSH}, which can easily accommodate lexical and budget requirements. Such methods optimize an integer programming objective, a well-known, NP-hard problem \cite{clarke2008global} requiring worst-case exponential computation.\footnote{ILPs are exponential in $|V|$ if when selecting tokens \cite{clarke2008global}, and exponential in $|E|$ when selecting edges \cite{filippova2013overcoming}.} ILP-based techniques typically optimize with an off-the-shelf solver. 

At this time, sequence to sequence sentence compression methods \cite{filippova2015sentence} are unsuitable for query-focused applications, because such methods cannot enforce lexical or length requirements. This limitation might be reexamined in future work, by modifying or adapting new constrained generation techniques \cite{N18-1119,aaimh}.

\section{Additive compression}\label{s:system}

In this work, we present a new transition-based method for shortening sentences under lexical and length constraints, inspired by similar approaches in transition-based parsing \cite{nivre2003}. We describe our technique as additive compression because it constructs a shortening by \textit{growing} a forest in the dependency parse of a sentence. This approach differs from some prior work which
compresses sentences by \textit{pruning} subtrees \cite{Knight2000StatisticsBasedS,berg2011jointly,almeida2013fast,Filippova2015FastKS}. The chief advantage of additive compression is that it can construct constrained compressions in linear time, leading to lower latency than ILP-based methods (\S\ref{s:autoeval}). 

Additive compression assumes a boolean relevance model, which defines some initial compression, containing the user's query term. Many search interfaces (e.g.\ Figure \ref{f:qf}) require such constrained compressions.


\begin{figure}[h]
\includegraphics[width=8.2cm]{additive.pdf}
\caption{A stateful procedure (from left to right) produces the compression $\{$A,C,B,E$\}$. Each candidate $v_i$ is boxed. All rejected candidates $\neg C$ are unshaded.}
\label{f:walkthru}
\end{figure}

\subsection{Additive procedure}\label{s:formal}

Our method builds a compression by maintaining a state
$(C_i,P_i)$ where $C_i \subseteq S$ is a set of added candidates, $P_i  \subseteq S$ is a priority queue of vertexes, and $i$ indexes a timestep during compression. Figure \ref{f:walkthru} shows a step-by-step example. 

During initialization, we set $C_0 \gets Q$ and $P_0 \gets S \setminus Q$. Then, at each timestep, we pop some candidate $v_i =h(P)$ from the head of $P$ and evaluate $v_i$ for inclusion in $C_i$. (Neighbors of $C_i$ in $P$ get higher priority than non-neighbors in the queue. We break ties in left-to-right order, by sentence position.) If we accept $v_i$, then $C_i \gets C_i \cup v_i$. We discuss acceptance decisions in detail in \S\ref{s:transition}. We continue adding vertexes to $C$ until either $P$ is empty or $C_i$ is $b$ characters long.\footnote{We linearize $C$ by left-to-right vertex position in $S$, which is common for English-language compression.} The appendix includes a formal algorithm. We never accept $v_i$ if the length of $C_i \cup v_i$ is more than $b$, respecting the character budget; we respect $Q$ because $C_0 \gets Q$. 

Our additive framework is linear in the token length of $S$ because $P_0  \gets S \setminus Q$, and because we pop and evaluate 1 vertex from $P$ at each timestep.

\section{Evaluation}\label{s:autoeval}

We compare additive compression to a supervised ILP baseline in terms of latency, readability and token-level F1 score. F1 is the standard automatic evaluation metric for the compression task, used to measure how well each compression method can reconstruct known-good shortenings. Our method achieves higher F1 scores, higher automatic readability scores and 100X lower latency (Table \ref{t:results}) than ILP techniques. We evaluate the significance of each difference with bootstrap sampling \cite{D12-1091}. All differences are significant {\small $(p < 10^{-2})$}. 

\subsection{Synthetic constrained compression experiment}\label{s:constrained}

In order to evaluate different approaches to constrained compression, we require a dataset of sentences, constraints and known-good shortenings, which respect the constraints. This means we need tuples $(S, Q, b, C_g)$, where $C_g$ is a known-good compression of $S$ which respects $Q$ and $b$ (\S\ref{s:intro}).

To support large-scale automatic evaluation, we reinterpret a standard compression corpus \cite{filippova2013overcoming}
as a collection of input triples and constrained compressions. The original dataset contains pairs of sentences $S$ and compressions $C_g$, generated using headlines. For our experiment, we set $b$ equal to  the character length of $C_g$, and we sample some query $Q$ from the tokens in $C_g$. We sample $Q$ to mimic the observed distribution of query token lengths \cite{Jansen2000RealLR} and observed distribution of query part-of-speech tags \cite{Barr2008TheLS} from real-world searches.\footnote{See appendix for detailed discussion of query sampling.} Sampled queries are short sets of nouns, such as ``police, Syracuse'', ``NHS'' and ``Hughes, manager, QPR.'' By sampling queries and defining budgets in this manner, we create {199,152} training tuples and {\ahcomment{TODO}} test tuples, each of the form $(S,Q,b,C_g)$. We re-tokenize, parse and tag the dataset with Stanford CoreNLP 3.8.0 \cite{corenlp}.

\subsection{Implementation: ILP-based compression}\label{s:ilp}

We compare our system to a state-of-the-art, ILP-based method, presented in \citet{filippova2013overcoming}. This approach represents each edge in a syntax tree with a vector of binary features, then learns weights for each feature using a structured perceptron trained on a corpus of sentence--compression pairs. Learned weights are used to compute a global compression objective, subject to structural constraints which ensure the output is a valid tree. This approach can easily perform constrained compression: after training the model, we specify that output must contain the query and respect the character budget.

To our knowledge, a public implementation of this method does not exist. We reimplement from scratch using \citet{gurobi}, achieving a test-time, token-level F1 score of  0.690 on the unconstrained compression task, lower than the result reported by the original authors. There are some important differences between our reimplementation and the method reported in \citet{filippova2013overcoming}, largely resulting from differences in syntactic formalisms. We describe these differences in detail in the appendix.

\subsection{Implementation: additive compression}\label{s:transition}

Additive compression accepts or rejects some candidate vertex $v_i$ at each timestep $i$. 
We learn such decisions using a corpus of tuples $(S,Q,b,C_g)$ (\S\ref{s:constrained}). Given such a tuple we can construct a unique oracle compression path by $(1)$ initializing $C_0$ with $Q$, $(2)$ choosing $v_i = h(P)$ at each timestep, and $(3)$ adding $v_i$ to $C_i$ iff $v_i \in C_g$. This procedure can reconstruct all $C_g$ in the original \citet{filippova2013overcoming} corpus. 

We then use a collection of such oracle paths to train a model of oracle inclusion decisions, ${p(y_i  = 1 | v_i, C_i, S_i)}$. We initially experimented with neural techniques, following the \citet{D14-1082} approach to transition-based parsing. But we found that feature-based, binary logistic regression achieved similar performance with much lower latency. This non-neural approach allows our method to be applied in fields like social science and journalism, where expensive graphics hardware is not available. The features in our model fall into 3 classes.

\textbf{Local features} describe the properties of the edge $(u,v)$ between $v_i \in P$ and $u \in C_i$. We use feature function from \citet{filippova2013overcoming}, described in detail in the appendix. This allows us to compare the performance of our local additive model with a global ILP objective (Table \ref{t:results}), which uses the same feature set.\footnote{If $v_i$ is disconnected from $C_i$, as in step 3 of Figure \ref{f:walkthru}, no local features are used.}

\textbf{Overall features} represent the relationship between $v_i$ and the overall compression $C_i$. Overall features include information such as the position of $v_i$ in the sentence, relative to the right-most and left-most vertex in $C_i$, or the fraction of the character budget used so far during addition. Such  features allow the feature-based model to reason about which sort of vertexes should be added to compressions in general. For instance, global features allow the model to learn if the oracle tends to add tokens to the right or left of $C_i$.

\textbf{Interaction features} are formed by crossing all global features with information about if $u$ governs $v_i$, if $v_i$ governs $u$ or if there is no edge $(u,v_i)$ in the parse.

The appendix contains additional details regarding model tuning and implementation. 

\subsection{Importance and readability evaluation}\label{s:readabilityinformativeness}

Researchers often use human judgements of \textit{importance} and \textit{readability} to evaluate compression techniques \cite{Knight2000StatisticsBasedS,clarke2008global,filippova2015sentence}. In our constrained compression setting, $Q$ determines the ``important'' information from $S$. Thus, human importance evaluations are inappropriate.

We use the automated SLOR metric \cite{lau2015unsupervised} to check the readability of compressions. Prior work shows that this metric correlates with human readability judgements for the compression task \cite{kannConl}. SLOR normalizes the probability of a token sequence assigned from a language model, by adjusting for both the probability of the individual unigrams in the sentence and for the sentence length.\footnote{Longer sentences are always less probable than shorter sentences; rarer words make a sequence less probable.} Based on SLOR scores, our method might produce slightly more readable compressions (Table \ref{t:results}). The appendix details our implementation of SLOR. 

\subsection{Latency evaluation}\label{s:costs}

Unlike ILPs, our method is linear instead of exponential in the worst-case (\S\ref{s:formal}). We test the real latency advantage from such theoretical gains by compressing 10,000 sentences sampled with replacement from the test set, and observing the wall clock speed. We repeat this experiment with our additive compression approach. Our method is \speedup faster (Table \ref{t:results}).The appendix contains additional computational details.

In user-facing applications such latency gains are non-trivial \cite{Nielsen,heerschei,Liu2014TheEO}: compressing even 10 sentences with an ILP (e.g.\ for 10 query results) takes more than a second. Additive compression finishes in under a tenth of a second. 

\begin{table}[]
\begin{tabular}{lccc}
\centering
Approach & F1 & SLOR &  seconds / sentence  \\ \hline
ILP&{\small 0.000}&{\small 0.000}&{\small 0.105} ({\small 0.053})\\
Rand.&{\small 0.493}&{\small 0.093}&{\small 0.001} ({\small 0.001})\\
\textbf{Additive}&\textbf{\small 0.845}&\textbf{\small 1.153}&\textbf{\small 0.008} (\textbf{\small 0.003})\\\end{tabular}
\caption{Test F1 scores, SLOR scores and average latency for three compression methods, for the constrained compression task. The difference between all metrics is statistically significant {\small $(p < 10^{-2})$}. Additive compression which accepts or rejects randomly (Rand.) at the marginal training-time acceptance rate achieves F1 = 0.490, a lower-bound for the task. \ahcomment{why only show sigma for some of these?}}
\label{t:results}
\end{table}

\section{Conclusion}

Low-latency, query-based compression is crucial for search applications. We introduce a new additive technique which is much \speedup faster than an ILP baseline, and which better reconstructs known-good constrained compressions. 

%Finally, some shortened sentences will modify the meaning of a sentence. Identifying such cases is a special case of the unsolved textual entailment problem \cite{snli_bowman,Pavlick2016SoCalledNA,linzencompression}. In the future, we plan to apply entailment research to the compression task.  

% ack => Katie, Javier, NLP reading group! 

%\appendix

\section{Appendix}

\subsection{Algorithm}
We formally present the \textsc{vertex addition} compression algorithm, using notation defined in the paper. $\ell$ linearizes a vertex set, based on left-to-right position in $S$; $|P|$ indicates the number of tokens in the priority queue.

\begin{algorithm}[]
\SetKwInOut{Input}{input}
\SetAlgoLined
\Input{$s=(V,E)$, $Q \subseteq V$, $b \in \mathbb{R^+}$}
 $ C \gets Q;  P \gets V \setminus Q$; \\
 \While{ $\ell(C) < b $ and $ |P| > 0 $   }{
  $v \gets \text{pop}(P)$; \\
  \If{$p(y=1) > .5$ and $\ell(C \cup \{v\}) \leq b$}{$C \gets C \cup \{v\}$}
 }
\KwRet{$\ell(C)$}
 \caption{\textsc{vertex addition}}
\end{algorithm}\label{a:algo}

\subsection{Reimplementation of \citet{filippova2013overcoming}}

In this work, we reimplement the method of \citet{filippova2013overcoming}, who in turn implement a method partially described in \citet{filippova2008dependency}.  There are inevitable discrepancies between our implementation and the methods described in these two prior papers.  

Some discrepancies arise from differences in syntactic formalisms, following from our choice to implement using the well-known Universal Dependencies (UD) v1 framework \cite{Nivre2016UniversalDV}. Most importantly, prior work implements a tree transformation as a preprocessing step. Some steps of this transform are incompatible with UD,\footnote{e.g.\ the tree transformation from prior work assumes PPs are headed by prepositions, which is not true in UD} and other steps are no longer necessary given changes in dependency representation. We make the following modifications.

In prior work, \citet{filippova2013overcoming} augment edge labels in the original dependency parse during transformation. We implement this step by assigning edge labels using the enhanced dependencies representation in CoreNLP \cite{Schuster2016EnhancedEU}. The augmented modifiers and augmented conjuncts from enhanced dependencies are very similar to the augmented edge labels from prior work.

\citet{filippova2013overcoming} also transform trees by adding an edge between the root node and all verbs in a sentence. (This step ensures that subclauses can be removed from parse trees, and then merged together to create a compression from different clauses of a sentence.) During development, we found that replicating this transform literally (i.e. only adding edges from the original root to all tokens tagged as verbs) made it impossible for the ILP to recreate some of the gold compressions in the dataset. (We suspect that this is because our part-of-speech tagger and the original part-of-speech tagger employed in \citet{filippova2013overcoming} sometimes return different part-of-speech tags.) Our tree transform therefore adds an edge between the root node and \textit{all} tokens in a sentence. With this change, it is always possible for the ILP to return the gold compression.

Other differences in implementation are not related to the tree transform.

\begin{itemize}
\item{We use \citet{gurobi} (v8) to solve the liner program. \citet{filippova2008dependency} report using LPsolve.\footnote{\url{http://
sourceforge.net/projects/lpsolve}}}
\item{We exclude a syntactic constraint based on the \rdep{sc} relation, which is not included in UD.}
\item{In Table 1 of their original paper, \citet{filippova2013overcoming} provide an overview of the syntactic, structural, semantic and lexical features in their model. We implement every feature explicitly described in their work, except where otherwise noted (e.g. syntactic feature not compatible with UD). However, additional features included in their model (but not explicily described in print) almost certainly affect performance.}
\end{itemize}

Finally, we assess convergence of the ILP by examining the validation F1 score on the traditional sentence compression task after each pass through the training data. We observe that F1 increases for each of five passes through the training data, and then decreases slightly (drops by $10^{-3}$ points). We thus terminate training after five epochs. Where the original authors train on only 100,000 sentences, we use the full training set to enable are more fair comparison with additive compression, which trains on all available data.

\subsection{Implementation of SLOR}

We use the SLOR function to measure the readability of the shortened sentences produced by each compression system. Following \cite{lau2015unsupervised}, we define the function as 

\begin{equation}
\text{SLOR}=\frac{\text{log}P_m(\xi) - \text{log}P_u(\xi)}{|\xi|}
\end{equation}

where $\xi$ is a sequence of words, $P_u(\xi)$ is the unigram probability of this sequence of words and $P_m(\xi)$ is the probability of the sequence, assigned by a language model.  $|\xi|$ is the length (in tokens) of the sentence.

We use a 3-gram language model trained on the \citet{filippova2013overcoming} corpus. We implement with KenLM \cite{Heafield-kenlm}.

\subsection{Latency evaluation}
To measure latency, for each technique, we sample 100,000 sentences with replacement from the test test. We observe the mean time per sentence using Python's built in \textit{timeit} module. Code for extracting edge features is shared across both methods, to ensure that each technique incurs very similar latency penalties for edge feature extraction. 

\subsection{Query sampling}
We sample queries for our synthetic constrained compression experiment to mimic real-world searches: the distribution of query token lengths and the distribution of query part-of-speech tags employed in our experiment closely match empirical distributions observed in real search \cite{Jansen2000RealLR,Barr2008TheLS}. To create queries, for each sentence in our corpus we: (1) sample a query token length in proportion to the real-world distribution over query token lengths (2) sample a proper or common noun in proportion to the real-world distribution over proper and improper nouns in queries, (3) add a noun or proper noun from the gold compression $C_g$ (based on step 2) to the set $\{Q\}$, if such a token exists (4) repeat steps 2 and 3 until $\{Q\}$ is the length specified in step 1. We exclude sentences in cases where the gold compression $C_g$ does not contain enough nouns to fill $\{Q\}$ to the desired length.

\bibliography{abe}
\bibliographystyle{acl_natbib}

\bibliography{abe}
\bibliographystyle{acl_natbib}

\end{document}

%\ahcomment{Sort of hard to tell what formalism F and A use. I thnk it is stanford, but they don't come out and say it.They cite Nivre's book which references the malt parser which seems to use stanford deps. but I don't see mention of the ``in" relation referenced in F and A paper in a guide to stanford deps. Writing around it.}

% other ideas... 
 
%\section{Computational experiments part 2: investigating properties of q,s,r compression}

%\ahcomment{include?}

%\ahcomment{only outline / notes here}

%\begin{enumerate}
%\item{q = a list of 1 to 3 NER}
%\item{r = random}
%\item{What is the size of the minimum compression?}
%\item{Reachability by budget by position of q in syntax tree. (If q is more than one entity then how the entities are dispersed across the tree probably matters a bunch too).}%
%\item{Hang on. reachability == min compression, eh? if min compression $>$ b, it is clearly bad.}
%\item{Avoid computational waste w/ grammar.  Examine: ops you never have to worry about if you prune a branch v. dependency type deletion endorsement rate. Some ops get rid of lots of tokens w/ very high probability of deletion endorsements: e.g. parataxis (a great op!). By contrast: pruning a noun subj destroys acceptability and usually does not delete many tokens. Not worth the risk!}
%\item{What is the empirical number of ops (i.e. decisions you have to make about pruning) if you greedily drop branches but never drop if the single op probability is less than $p$? My guess is you can make this problem way, way, simpler than implied by exponential formulation. Is it really quadratic?}
%\item{Distribution of number of ops used for different q and r: when choosing ops at random? when choosing greedily? When pruning $\propto$ p(endorsement)?}
%\item{Other stuff: min compression, reachability, operations saved w/ big prunes? position of query in the tree?}
%\end{enumerate}
